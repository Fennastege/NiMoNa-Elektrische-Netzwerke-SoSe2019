\documentclass[11pt,a4paper,titlepage]{article}
\usepackage[utf8]{inputenc}
\usepackage[T1]{fontenc}
\usepackage{ngerman}
\usepackage{lmodern}
\usepackage{graphicx}
\usepackage{url}
\usepackage{color}
\usepackage{transparent}
\usepackage[add-decimal-zero, add-integer-zero, separate-uncertainty=true]{siunitx}
\usepackage{float}
\usepackage{amsmath}
\usepackage{mathtools}
\usepackage{hyperref}
\usepackage{footmisc}
\usepackage{wasysym}
\graphicspath{{PDF/M5}}
 \mathcode`\,="013B

\sisetup{
	locale = DE,
	per-mode = fraction
}

\begin{document}
\section{Erweitertes Kuramoto-Modell}
\begin{equation}
	\frac{d^2}{dt^2}\phi_j=P_j-\alpha\frac{d}{dt}\phi_j-\sum_iK_{ij}\sin{(\phi_i-\phi_j)}
\end{equation}
\section{RK4}
\begin{align}
	\nonumber
	&k_1=f(t_n, \ x_n) \\
	\nonumber
	&k_2=f(t_n+\frac{h}{2}, \ x_n+\frac{h}{2}k_1) \\
	\nonumber
	&k_3=f(t_n+\frac{h}{2}, \ x_n+\frac{h}{2}k_2) \\
	\nonumber
	&k_4=f(t_n+h, \ x_n+hk_3) \\
\end{align}
\end{document} 