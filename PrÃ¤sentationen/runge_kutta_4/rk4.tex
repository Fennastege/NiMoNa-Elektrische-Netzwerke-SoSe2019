\documentclass[]{scrartcl}

\begin{document}

\title{Runge-Kutta-4}
\maketitle

\begin{enumerate}
	\item Was wollen wir machen?
	\begin{itemize}
		\item Lösen einer DGL 1. Ordnung: $\dot{x}=f(x,t)$
		\item numerisch, mit Schrittweise $h$
		\item möglichst geringer Fehler
		\item angemessener Rechenaufwand
	\end{itemize}
	\item Wie wollen wir das Erreichen?
	\begin{itemize}
		\item Ansatz:\begin{equation}
			x_{n+1}= x_n+h\cdot\sum_{i=1}^{m}c_i\,k_i,
		\end{equation}
		wo $k_i=f(x_n+h\sum_{j=1}^{i-1}\beta_{ij}k_j,t_n+\alpha_ih)$
		\item Aber was ist $c_i,\,\alpha_i,\,\beta_{ij}$?
		\item Dazu Taylorn von Steigung zwischen $(x(t_n),t_n)$ und $(x(t_n+h),t_n+h)$ bis zur Ordnung $m$\\
		$\rightarrow$ Fehlerordnung $\mathcal{O}(h^{m+1})$
		\item Ansatz Taylorn\\
		$\rightarrow$ Koeffizientenvergleich (ab 2. Ordnung unbestimmtes Gleichungssystem)
		\item Butcher-Tabellen:
		\begin{tabbing}
			\quad$\Rightarrow$ \=1. Ordnung: Eulerverfahren: $x_{n+1}=x_n+h\cdot f(x_n, t_n)$\\
			\>2. Ordnung: Heunverfahren: $x_{n+1}=x_n+\frac{h}{2}\left[f(x_n,t_n)+f(x_n+h\cdot f(x_n,t_n),f(x_n,t_n))\right]$\\
			\>4. Ordnung: Runge-Kutte-4:
		\end{tabbing}
		\begin{equation}
			x_{n+1}=x_n+\frac{h}{6}\cdot (k_1+2k_2+2k_3+k_4)
		\end{equation} 
	\begin{center}
		\begin{tabular}{c|cccc}
			0&&&&\\
			1/2&1/2&&&\\
			1/2&0&1/2&&\\
			1&0&0&1&\\ \hline
			&1/6&1/3&1/3&1/6
		\end{tabular}
	\end{center}	
	\end{itemize}
	\item Warum 4. Ordnung?
	\begin{itemize}
		\item Butcher-Barriere: Für $m\geq 5$ zu großer Rechenaufwand im Verhältnis zur Genauigkeit.
	\end{itemize}
\end{enumerate}
\end{document}
